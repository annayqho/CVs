%%%%%%%%%%%%%%%%%%%%%%%%%%%%%%%%%%%%%%%%%
% Medium Length Professional CV
% LaTeX Template
% Version 2.0 (8/5/13)
%
% This template has been downloaded from:
% http://www.LaTeXTemplates.com
%
% Original author:
% Trey Hunner (http://www.treyhunner.com/)
%
% Important note:
% This template requires the resume.cls file to be in the same directory as the
% .tex file. The resume.cls file provides the resume style used for structuring the
% document.
%
%%%%%%%%%%%%%%%%%%%%%%%%%%%%%%%%%%%%%%%%%

%----------------------------------------------------------------------------------------
%	PACKAGES AND OTHER DOCUMENT CONFIGURATIONS
%----------------------------------------------------------------------------------------

\documentclass{resume} % Use the custom resume.cls style

\usepackage[left=0.75in,top=0.6in,right=0.75in,bottom=0.6in]{geometry} % Document margins
\usepackage{hyperref}

\name{Anna Ho} % Your name
%\address{Max-Planck-Institut f{\"u}r Astro¨nomie \\ K{\"o}nigstuhl 17, 69117 Heidelberg, Germany} % Your address
\address{ayho@caltech.edu \\ annaho.net} % Your phone number and email

\begin{document}

%----------------------------------------------------------------------------------------
%	EDUCATION SECTION
%----------------------------------------------------------------------------------------

\begin{rSection}{Education}

{\bf California Institute of Technology (Caltech)}, Pasadena CA
\begin{itemize}
\item Ph.D., Astrophysics \hfill expected {\em June 2020} 
\end{itemize}

{\bf Massachusetts Institute of Technology (MIT)}, Cambridge MA
\begin{itemize}
\item B.S., Physics \hfill {\em June 2014} 
\end{itemize}

\end{rSection}

%----------------------------------------------------------------------------------------
%	RESEARCH EXPERIENCE SECTION
%----------------------------------------------------------------------------------------

\begin{rSection}{Research Experience}

\begin{rSubsection}{Max Planck Institute for Astronomy}{September 2014-present}{Fulbright Scholarship}{Heidelberg, Germany}
\item Advisor: Prof. Hans-Walter Rix
\item Project Title: ``Survey Cross-Calibration using {\em The Cannon}: {\em APOGEE} Labels from {\em LAMOST} Spectra"
\end{rSubsection}

\begin{rSubsection}{National Radio Astronomy Observatory}{Summer 2012, Summer 2013}{NSF Research Experiences for Undergraduates (REU) Program}{Charlottesville, VA}
\item Advisor: Dr. Scott Ransom
\item 2012 Project Title: ``A New Method for Measuring the Rotation Measures of Millisecond Pulsars in the Globular Cluster Terzan 5"
\item 2013 Project Title: ``Globular Cluster Millisecond Pulsars as a Unique Probe of the Galactic Magnetic Field"
\end{rSubsection}

\end{rSection}

%----------------------------------------------------------------------------------------
%	OBSERVING TIME SECTION
%----------------------------------------------------------------------------------------

\begin{rSection}{Telescope Time}

\begin{rSubsection}{EVLA}{}{}{}
\item PI: ``Monitoring the Pulsed and Continuum Fluxes of Eclipsing Binary Pulsar Terzan 5A" \\
4 hours, Priority B \hfill {\em February 2014}
\item ``Detailed Magnetic Field Mapping of the Supernova Remnant 3C 58" \\
8 hours, Director's Discretionary Time, PI: Diana Powell \hfill {\em August 2013}
\end{rSubsection}

\end{rSection}

%----------------------------------------------------------------------------------------
%	ORAL AND POSTER PRESENTATIONS SECTION
%----------------------------------------------------------------------------------------

\begin{rSection}{Oral and Poster Presentations}
``Survey Cross-Calibration Using The Cannon: APOGEE Labels from LAMOST Spectra"
\begin{itemize}
\item
Talk, SDSS-IV Collaboration Meeting \hfill {\em July 2015} \\
Instituto de Física Teórica IFT UAM-CSIC, Madrid, Spain
\item
Talk, The Local Group Astrostatistics Conference \hfill {\em June 2015} \\
University of Michigan, Ann Arbor, USA
\end{itemize}

``The Cannon: A New Data-Driven Method for Retrieving Stellar Parameters and Abundances"
\begin{itemize}
\item
Talk, MPIA-AIP Milky Way \& Local Volume Meeting  \hfill {\em November 2014} \\
Institute for Astrophysics Potsdam (AIP), Potsdam, Germany
\end{itemize}

``Rotation Measures of Globular Cluster Pulsars as a Unique Probe of the Galactic Magnetic Field"
\begin{itemize}
\item
Talk, Max Planck Institute for Astronomy,
Heidelberg, Germany \hfill {\em October 2014}
\item
Poster, American Astronomical Society 223rd Meeting, 
National Harbor, MD \hfill {\em January 2014}
\item
Poster, Conference of Research Experiences for Undergraduates, \hfill {\em October 2013}\\
Council on Undergraduate Research, Arlington, VA 
\item
Talk, National Radio Astronomy Observatory, 
Charlottesville, VA \hfill {\em August 2013} 
\end{itemize}

``A New Method for Measuring the Rotation Measures of Millisecond Pulsars in the Globular Cluster Terzan 5" 
\begin{itemize}
\item
Poster, American Astronomical Society 221st Meeting, 
Long Beach, CA \hfill {\em January 2013}
\end{itemize}

``Studies of Millisecond Pulsars in the Globular Cluster Terzan 5" 
\begin{itemize}
\item
Talk, National Radio Astronomy Observatory, 
Charlottesville, VA \hfill {\em August 2013} 
\end{itemize}

\end{rSection}

%----------------------------------------------------------------------------------------
%	HONORS AND AWARDS
%----------------------------------------------------------------------------------------

\begin{rSection}{Fellowships and Awards}

\begin{itemize}
\item
NSF Graduate Research Fellowship \hfill {\em 2014-17}
\item
Fulbright Scholarship \hfill {\em 2014-15}
\item
Ida M. Green Fellowship, MIT \hfill {\em 2014} 
\item
Ford Foundation Fellowship, Honorable Mention \hfill {\em 2014} 
\item
Karl Taylor Compton Prize, MIT \hfill {\em 2014} 
\item
Chambliss Astronomy Achievement Student Awards, Honorable Mention \hfill {\em 2014}
\item
First Place, Dewitt Wallace Prize for Science Writing for the Public, MIT \hfill {\em 2013} 
\item
Burchard Scholars Program, MIT \hfill {\em 2012} \\
\end{itemize}

\end{rSection}

%----------------------------------------------------------------------------------------
%	TECHING AND OUTREACH EXPERIENCE SECTION
%----------------------------------------------------------------------------------------

\begin{rSection}{Teaching, Public Outreach, and Policy Work}

\begin{rSubsection}{Congressional Visits Day}{March 2014}{}{Washington, DC}
\item Attended briefings about the federal budget process
\item Set up and led meetings with Congressional Staff to advocate for federal funding for scientific research
\end{rSubsection}

\begin{rSubsection}{AAS Astronomy Ambassadors Workshop}{January 2014}{American Astronomical Society 223rd Meeting}{National Harbor, MD}
\item Selected for a two-day workshop on doing effective public outreach 
\end{rSubsection}

\begin{rSubsection}{MIT Educational Studies Program}{Fall 2010-Spring 2014}{Teacher}{Cambridge, MA}
\item Designed and taught 12 different science classes for over 500 middle- and high-school students
\end{rSubsection}

\begin{rSubsection}{MIT Admissions Blogger}{Fall 2010-Spring 2014}{MIT Admissions Office}{Cambridge, MA}
\item Wrote weekly entries about MIT life, read by over 7,000 people daily
\item Corresponded with prospective students through e-mail and webcasts
\end{rSubsection}

\begin{rSubsection}{Course Assistant}{Spring 2013}{MIT Physics Department}{Cambridge, MA}
\item Wrote lecture notes in LaTeX for the undergraduate Quantum I and Quantum II courses
\item Graded weekly problem sets for the undergraduate Quantum I course
\end{rSubsection}

\begin{rSubsection}{McCormick Public Observatory}{Summer 2012, Summer 2013}{Public speaker and volunteer}{Charlottesville, VA}
\item Organized a volunteering program for National Radio Astronomy Observatory summer students
\item Gave regular public talks at the observatory
\end{rSubsection}

\begin{rSubsection}{MIT Four Weeks For America Program, Teach For America}{January 2011}{Teaching Assistant}{Pueblo Pintado Navajo Reservation, NM}
\item Created new modules for the high school math curriculum
\item Tutored and mentored students
\end{rSubsection}

\end{rSection}

%----------------------------------------------------------------------------------------
%	PUBLICATIONS
%----------------------------------------------------------------------------------------

\begin{rSection}{Publications}

Ness, M., Hogg, D.W., Rix, H-W., Ho, A. Y. Q., Zasowski, G. \href{http://arxiv.org/abs/1501.07604}{The Cannon: A data-driven approach to stellar label determination (arXiv:1501.07604)}, {\em Astrophys. J.} 

\end{rSection}

%----------------------------------------------------------------------------------------

\end{document}
