%%%%%%%%%%%%%%%%%%%%%%%%%%%%%%%%%%%%%%%%%
% Medium Length Professional CV
% LaTeX Template
% Version 2.0 (8/5/13)
%
% This template has been downloaded from:
% http://www.LaTeXTemplates.com
%
% Original author:
% Trey Hunner (http://www.treyhunner.com/)
%
% Important note:
% This template requires the resume.cls file to be in the same directory as the
% .tex file. The resume.cls file provides the resume style used for structuring
% the document.
%
%%%%%%%%%%%%%%%%%%%%%%%%%%%%%%%%%%%%%%%%%

%------------------------------------------------------------------------------
%	PACKAGES AND OTHER DOCUMENT CONFIGURATIONS
%------------------------------------------------------------------------------

\documentclass{resume} % Use the custom resume.cls style

% document margins
\usepackage[left=0.75in,top=0.6in,right=0.75in,bottom=0.6in]{geometry} 
\usepackage{hyperref}

\name{Anna Ho} % Your name
\address{ah@astro.caltech.edu} % Your phone number and email
\address{Caltech, MC 249-17, 1200 E. California Blvd., Pasadena CA 91125}

\begin{document}

%------------------------------------------------------------------------------
%	EDUCATION SECTION
%------------------------------------------------------------------------------

\begin{rSection}{Education}

{\bf California Institute of Technology (Caltech)}, Pasadena CA
\begin{itemize}
\item Ph.D., Astrophysics, 
  Advisor: Prof. Shrinivas Kulkarni \hfill expected {\em June 2020} 
\end{itemize}

\begin{itemize}
  \item M.S., Astrophysics \hfill {\em June 2017}
\end{itemize}

{\bf Massachusetts Institute of Technology (MIT)}, Cambridge MA
\begin{itemize}
\item B.S., Physics \hfill {\em June 2014} 
\end{itemize}

\end{rSection}

%------------------------------------------------------------------------------
%   RESEARCH INTERESTS
%------------------------------------------------------------------------------

%\begin{rSection}{Research Interests}
%Relativistic explosions, time domain astronomy, radio interferometry,
%data-driven modeling, stellar spectroscopy, Milky Way structure and formation, 
%millisecond pulsars

%\end{rSection}

%------------------------------------------------------------------------------
%	HONORS AND AWARDS
%------------------------------------------------------------------------------

\begin{rSection}{Fellowships and Awards}

\begin{itemize}
\item
NSF Graduate Research Fellowship \hfill {\em 2014-19}
\item
Garmire Prize, Caltech (Division Award) \hfill {\em 2017}
\item
Fulbright Scholarship \hfill {\em 2014-15}
\item
Ida M. Green Fellowship, MIT (Departmental Award) \hfill {\em 2014} 
\item
Ford Foundation Fellowship, Honorable Mention \hfill {\em 2014} 
\item
Karl Taylor Compton Prize, MIT (University Award) \hfill {\em 2014} 
\item
Chambliss Astronomy Achievement Student Awards, Honorable Mention \hfill {\em 2014}
\item
First Place, Dewitt Wallace Prize for Science Writing for the Public, MIT \hfill {\em 2013} 
\end{itemize}

\end{rSection}

%------------------------------------------------------------------------------
%	INVITED AND CONTRIBUTED TALKS
%------------------------------------------------------------------------------

\begin{rSection}{Invited and Contributed Talks}

``SMA Observations of AT2018cow: A Prototype for Millimeter Time-Domain Astronomy''
\begin{itemize}
  \item
    (Contributed) SMA Seminar, Harvard-Smithsonian CfA, Cambridge, MA
    \hfill {\em April 2019}
\end{itemize}

``The Death Throes of a Stripped Massive Star''
\begin{itemize}
  \item
    (Contributed) UC Berkeley Department Lunch Talk, Berkeley, CA
    \hfill {\em April 2019}
  \item
    (Contributed) Stars and Planets Seminar, Harvard-Smithsonian CfA, Cambridge, MA
    \hfill {\em April 2019}
  \item
    (Invited) Brown Bag Lunch, MIT, Cambridge, MA
    \hfill {\em April 2019}
  \item
    (Contributed) STScI Spring Symposium, Baltimore, MD
    \hfill {\em April 2019}
\end{itemize}

``Watching The Cow Shock Its Environment: The Millimeter-Wavelength Perspective''
\begin{itemize}
  \item
    (Invited) Talk (Press Panel) at the AAS Winter Meeting, Seattle, WA
    \hfill {\em January 2019}
\end{itemize}

``ZTF18abukavn (AT2018gep): a luminous, rapidly rising, high-velocity Ic-BL supernova''
\begin{itemize}
  \item
    (Contributed) ZTF-Theory Network Meeting, KITP, Santa Barbara, CA
    \hfill {\em December 2018}
\end{itemize}

``AT2018cow: a luminous millimeter transient''
\begin{itemize}
  \item
    (Contributed) ZTF-Theory Network Meeting, KITP, Santa Barbara, CA
    \hfill {\em July 2018}
\end{itemize}

``Dirty Fireballs and Orphan Afterglows: A Broader Landscape of Relativistic Explosions with ZTF''
\begin{itemize}
  \item
    (Contributed) GROWTH Annual Meeting, Milwaukee, WI
    \hfill {\em October 2017}
\end{itemize}

``\emph{The Cannon:} Data-Driven Spectral Modeling in the Era of Large Stellar Surveys''
\begin{itemize}
  \item
    (Contributed) National Radio Astronomy Observatory Lunch Seminar 
    \hfill {\em June 2016} \\
    National Radio Astronomy Observatory, Socorro, NM
  \item
  (Invited) Gemini Observatory Workshop, La Serena, Chile \hfill {\em March 2016}
\end{itemize}
``Using \emph{The Cannon} to Exploit the Overlap Between Kepler \& APOGEE''
\begin{itemize}
  \item
    (Contributed) Boutiques \& Experiments Conference, Caltech, Pasadena, CA \hfill {\em August 2015}
\end{itemize}
``Survey Cross-Calibration Using The Cannon: APOGEE Labels from LAMOST Spectra"
\begin{itemize}
\item
(Contributed) SDSS-IV Collaboration Meeting \hfill {\em July 2015} \\
Instituto de Física Teórica IFT UAM-CSIC, Madrid, Spain
\item
(Contributed) The Local Group Astrostatistics Conference \hfill {\em June 2015} \\
University of Michigan, Ann Arbor, USA
\end{itemize}

``The Cannon: A New Data-Driven Method for Retrieving Stellar Parameters and Abundances"
\begin{itemize}
\item
(Contributed) MPIA-AIP Milky Way \& Local Volume Meeting  \hfill {\em November 2014} \\
Institute for Astrophysics Potsdam (AIP), Potsdam, Germany
\end{itemize}

``Rotation Measures of Globular Cluster Pulsars as a Unique Probe of the Galactic Magnetic Field"
\begin{itemize}
\item
(Contributed) Max Planck Institute for Astronomy,
Heidelberg, Germany \hfill {\em October 2014}
\item
(Contributed) National Radio Astronomy Observatory, 
Charlottesville, VA \hfill {\em August 2013} 
\end{itemize}


``Studies of Millisecond Pulsars in the Globular Cluster Terzan 5" 
\begin{itemize}
\item
(Contributed) National Radio Astronomy Observatory, 
Charlottesville, VA \hfill {\em August 2012} 
\end{itemize}

\end{rSection}

%----------------------------------------------------------------------------------------
%   POSTERS
%----------------------------------------------------------------------------------------

\begin{rSection}{Posters}

``AT2018cow: A Luminous Millimeter Transient''
\begin{itemize}
  \item
    American Astronomical Society 233rd Winter Meeting, Seattle, WA
    \hfill {\em January 2019}\\
\end{itemize}

``Rotation Measures of Globular Cluster Pulsars as a Unique Probe of the Galactic Magnetic Field"
\begin{itemize}
\item
American Astronomical Society 223rd Meeting, 
National Harbor, MD \hfill {\em January 2014}
\item
Conference of Research Experiences for Undergraduates, \hfill {\em October 2013}\\
Council on Undergraduate Research, Arlington, VA 
\end{itemize}

``Survey Cross-Calibration Using \emph{The Cannon}:
LAMOST Labels on the APOGEE scale''
\begin{itemize}
  \item
    Frontiers of Stellar Spectroscopy in the Local Group and Beyond,
    \hfill {\em April 2015}\\
    Max Planck Institute for Astronomy, Heidelberg, Germany 

\end{itemize}

``Rotation Measures of Globular Cluster Pulsars as a Unique Probe of the Galactic Magnetic Field"
\begin{itemize}
\item
American Astronomical Society 223rd Meeting, 
National Harbor, MD \hfill {\em January 2014}
\item
Conference of Research Experiences for Undergraduates, \hfill {\em October 2013}\\
Council on Undergraduate Research, Arlington, VA 
\end{itemize}

``A New Method for Measuring the Rotation Measures of Millisecond Pulsars in the Globular Cluster Terzan 5" 
\begin{itemize}
\item
American Astronomical Society 221st Meeting, 
Long Beach, CA \hfill {\em January 2013}
\end{itemize}

\end{rSection}

%----------------------------------------------------------------------------------------
%   WORKSHOPS
%----------------------------------------------------------------------------------------

\begin{rSection}{Workshops}

  \begin{itemize}
    \item
      \emph{Attendee,} LSST Winter School: Machine Learning, Data Visualization, Model Fitting \\
      Caltech, Pasadena CA \hfill January 2017

    \item
      \emph{Attendee,} NRAO Summer School, Socorro, NM \hfill June 2016 
    \item 
      \emph{Invited Instructor,} Gemini Observatory Workshop: \hfill March 2016 \\
      Data Driven Modeling of Spectra using \emph{The Cannon}, Gemini Observatory, La Serena, Chile 

  \end{itemize}


\end{rSection}


%----------------------------------------------------------------------------------------
%	OBSERVING TIME SECTION
%----------------------------------------------------------------------------------------

\begin{rSection}{Accepted Proposals, Telescope Time}

\begin{rSubsection}{ALMA}{}{}{}
\item PI: ``AT2018cow: the poster-child relativistic explosion \\
  for high-frequency time-domain astronomy'' \\
  2.6 hours at highest priority \hfill {\em June 2018}
\end{rSubsection}

\begin{rSubsection}{SMA}{}{}{}
\item PI: ``SMA Monitoring of the Rare Relativistic Supernova AT2018cow'' \\
  80 hours \hfill {\em June 2018}
\end{rSubsection}

\begin{rSubsection}{VLBA}{}{}{}
\item PI: ``Confirmation of superluminal motion for \\
  rare relativistic supernova AT2018cow''\\
  12 hours at Priority A \hfill {\em June 2018}
\end{rSubsection}

\begin{rSubsection}{EVLA}{}{}{}
\item PI: ``MAXI 170808A, A Short Soft X-ray Transient'' \\
  3 hours of DDT time at Priority C \hfill {\em August 2017}
\item PI: ``A Short Soft Transient from MAXI: Detection of a Dirty Fireball?'' \\
  3 hours of DDT time at Priority C \hfill {\em May 2017}
\item PI: ``Monitoring the Pulsed and Continuum Fluxes of Eclipsing Binary Pulsar Terzan 5A" \\
4 hours Priority B \hfill {\em February 2014}
\end{rSubsection}

\end{rSection}

%----------------------------------------------------------------------------------------
%	TEACHING AND OUTREACH EXPERIENCE SECTION
%----------------------------------------------------------------------------------------

\begin{rSection}{Teaching, Public Outreach, Science Policy}

  \begin{rSubsection}{Science and Engineering Policy At Caltech}{Fall 2017-present}
    {Vice President}{Pasadena, CA}
  \item Organize events and trips,
    lead lunch discussions on current events in science policy
  \end{rSubsection}

  \begin{rSubsection}{Caltech Astronomy Outreach}{Feb 2015-present}
    {Co-Coordinator}{Pasadena, CA}
  \item Run outreach evenings, give public talks, answer visitors' questions, 
    facilitate telescope viewing
  \end{rSubsection}

  \begin{rSubsection}{International Summer Symposium on Science and World Affairs}{Summer 2017}
      {Participant and Speaker}{Darmstadt, Germany}
    \item Selected to attend this annual international symposium
      %which aims to cultivate a group of young scientists working on international security issues.
    \item Gave a talk entitled ``Towards a Framework for Space Traffic Control''
  \end{rSubsection}

  \begin{rSubsection}
    {TA for Undergraduate Course, The Evolving Universe}{Spring 2016}{}{Pasadena, CA}
  \item Recognized as an ``outstanding TA'' by the Caltech registrar:
    ``Students described Anna as caring, considerate, and committed \ldots 
    as well as being extremely effective at explaining and summarizing the 
    course material. The sentiments in this quote were echoed by several other 
    students: ''She was consistently well-prepared for section, 
    gave really good notes, and did a really good job of explaining 
    potentially confusing material and clarifying misunderstandings. 
    She was very in-tune with the difficulties students were having 
    and did a very good job of resolving those difficulties.``
  \end{rSubsection}


  \begin{rSubsection}{TA for Graduate Course, Radio Astronomy}
    {Winter Term 2015-16}{}{Pasadena, CA}
  \item Graded problem sets, held office hours
  \end{rSubsection}

  \begin{rSubsection}
    {TA for Undergraduate Course, Basic Astronomy and the Galaxy}{Fall 2015}
    {}{Pasadena, CA}
  \item Graded problem sets, held office hours
  \end{rSubsection}

  \begin{rSubsection}{Teacher, Institute for Educational Advancement}
    {Fall 2016}{}{Pasadena, CA}
  \item Designed and taught a nine-week course on multiwavelength astronomy
    for gifted 7-12 year olds
  \end{rSubsection}

  \begin{rSubsection}
    {Haus der Astronomie: Center for Astronomy Education and Outreach}
    {Sept 2014-July 2015}{}{Heidelberg, Germany}
  \item Organized and taught a cosmology workshop for high school students
  \item Wrote \href{http://www.mpia.de/news/science/2015-03-biosignatures}
    {a press release} for the Max Planck Institute for Astronomy
  \item Wrote \href{http://www.universetoday.com/120820/distant-stellar-nurseries-this-time-in-high-definition/}
    {a blog post} for the UniverseToday news site

  \end{rSubsection}

\begin{rSubsection}{Congressional Visits Day}{March 2014}{}{Washington, DC}
\item Attended briefings about the federal budget process
\item Set up and led meetings with Congressional Staff to advocate 
  for federal funding for scientific research
\end{rSubsection}

\begin{rSubsection}{AAS Astronomy Ambassadors Workshop}{January 2014}
  {American Astronomical Society 223rd Meeting}{National Harbor, MD}
\item Selected for a two-day workshop on doing effective public outreach 
\end{rSubsection}

\begin{rSubsection}{MIT Educational Studies Program}{Fall 2010-Spring 2014}
  {Teacher}{Cambridge, MA}
\item Designed and taught 12 different science classes for over 500 middle- 
  and high-school students
\end{rSubsection}

\begin{rSubsection}{MIT Admissions Blogger}{Fall 2010-Spring 2014}
  {MIT Admissions Office}{Cambridge, MA}
\item Wrote weekly entries about MIT life, read by over 7,000 people daily
\item Corresponded with prospective students through e-mail and webcasts
\end{rSubsection}

\begin{rSubsection}{Course Assistant}{Spring 2013}
  {MIT Physics Department}{Cambridge, MA}
\item Wrote lecture notes in LaTeX for the undergraduate Quantum I 
  and Quantum II courses
\item Graded weekly problem sets for the undergraduate Quantum I course
\end{rSubsection}

\begin{rSubsection}{McCormick Public Observatory}{Summer 2012, Summer 2013}
  {Public speaker and volunteer}{Charlottesville, VA}
\item Organized a volunteering program for National Radio Astronomy Observatory 
  summer students
\item Gave regular public talks at the observatory
\end{rSubsection}

%\begin{rSubsection}{MIT Four Weeks For America Program, Teach For America}
%  {January 2011}{Teaching Assistant}{Pueblo Pintado Navajo Reservation, NM}
%\item Created new modules for the high school math curriculum
%\item Tutored and mentored students
%\end{rSubsection}

\end{rSection}

%----------------------------------------------------------------------------------------
%	PUBLICATIONS
%----------------------------------------------------------------------------------------

% \begin{rSection}{Publications (2 lead author, 6 co-author)}
% 
% \textbf{Ho, A.~Y.~Q.}, Rix, H.-W., Ness, M.~K., Hogg, D.~W., et al.\ 2016, 
% \emph{Masses and Ages for 230,000 LAMOST Giants,
% via Their Carbon and Nitrogen Abundances},
% ApJ submitted,
% \href{http://arxiv.org/abs/1609.03195}{(arXiv:1609.03195)}
% 
% \textbf{Ho, A.~Y.~Q.}, Ness, M.~K., Hogg, D.~W., et al.\ 2016, 
% \emph{Label Transfer from APOGEE to LAMOST: Precise Stellar Parameters for 
% 450,000 LAMOST Giants}, ApJ,
% \textbf{836}, 5
% \href{http://arxiv.org/abs/1602.00303}{(arXiv:1602.00303)}
% 
% Ting, Y.-S., Rix, H.-W., Conroy, C., \textbf{Ho, A.~Y.~Q.}, \& Lin, J.\ 2017, \href{https://arxiv.org/abs/1708.01758}{(arXiv:1708.01758)}
% 
% Blagorodnova, N., et al. (including {Ho, A. Y. Q.}) \ 2017,
% \emph{iPTF16fnl: a faint and fast tidal disruption event in an E+A galaxy},
% ApJ submitted,
% \href{https://arxiv.org/abs/1703.00965}{(arXiv:1703.00965)}
% 
% Ness, M., Hogg, D.W., Rix, H-W., \textbf{Ho, A. Y. Q.}, Zasowski, G. 2015,
% \emph{The Cannon: A data-driven approach to stellar label determination}, ApJ,
% \textbf{808}, 16 \href{http://arxiv.org/abs/1501.07604}{(arXiv:1501.07604)}
% 
% Casey, A.~R., Hogg, D.~W., Ness, M., Rix, H.-W., 
% \textbf{Ho, Anna Y. Q.}, Gilmore, G. \ 2016
% \emph{The Cannon 2: 
% A data-driven model of stellar spectra for detailed 
% chemical abundance analyses},
% ApJ submitted,
% \href{http://arxiv.org/abs/1603.03040}{(arXiv:1603.03040)}
% 
% Ness, M., Hogg, D.~W., Rix, H.-W., Martig, M., Pinsonneault, M. H.,
% \textbf{Ho, A. Y. Q.} \ 2016, 
% \emph{Spectroscopic Determination of Masses (and Implied Ages) for Red Giants},
% ApJ, \textbf{823}, 114 \href{http://arxiv.org/abs/1511.08204}
% {(arXiv:1511.08204)}
% 
% Hogg, D.~W., Casey, A.~R., Ness, M., Rix, H.-W., Foreman-Mackey, D., 
% Hasselquist, S., \textbf{Ho, Anna Y. Q.} et al.\ 2016,
% \emph{Chemical tagging can work: Identification of stellar 
% phase-space structures purely by chemical-abundance similarity},
% ApJ, \textbf{833}, 262
% \href{http://arxiv.org/abs/1601.05413}{(arXiv:1601.05413)}
% 
% \end{rSection}
% 
%----------------------------------------------------------------------------------------

\end{document}
