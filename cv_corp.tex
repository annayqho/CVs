%------------------------------------------------------------------------------
%   PACKAGES AND OTHER DOCUMENT CONFIGURATIONS
%------------------------------------------------------------------------------

\documentclass{resume} % Use the custom resume.cls style

% document margins
\usepackage[left=0.75in,top=0.6in,right=0.75in,bottom=0.6in]{geometry}
\usepackage{hyperref}

\name{Anna Ho} % Your name
\address{ah@astro.caltech.edu} % Your phone number and email
\address{Citizenship: US, UK}
\address{Caltech, MC 249-17, 1200 E. California Blvd., Pasadena CA 91125}

\begin{document}

%------------------------------------------------------------------------------
%   EDUCATION SECTION
%------------------------------------------------------------------------------

\begin{rSection}{Education}

{\bf California Institute of Technology (Caltech)}, Pasadena CA
\begin{itemize}
  \item Ph.D., Astrophysics \hfill \emph{expected June 2020}
\item Graduate coursework includes: 
  Physics of Measurement, Mathematical Methods of Physics,
  High-Energy Astrophysics, Astronomical Measurements and Instrumentation
\end{itemize}

{\bf Massachusetts Institute of Technology (MIT)}, Cambridge MA
\begin{itemize}
\item B.S., Physics, GPA 4.5/5.0 \hfill {\em June 2014}
\item Coursework includes: Differential Equations, Linear Algebra,
  Classical Mechanics II, Quantum Physics II-III,
  Modern Space Science and Engineering, Experimental Physics
\end{itemize}

\end{rSection}


%------------------------------------------------------------------------------
%	RESEARCH SECTION
%------------------------------------------------------------------------------

\begin{rSection}{Research Projects}
  \begin{rSubsection}{Data-Driven Modeling for Large Datasets of Spectra}
    {2014-17}{}{}
  \item 
    Full-time research as a graduate student at Caltech and a Fulbright
    Scholar at the Max Planck Institute for Astronomy in Heidelberg, Germany
  \item
    Wrote \href{https://github.com/annayqho/TheCannon}{open-source code} in Python to model large datasets of spectra
  \item
    On invitation, designed and taught a two-day Python workshop on modeling
    specta, at Gemini Observatory in La Serena, Chile.
    Audience include 13 graduate students, post docs, and faculty.
  \item
    Presented at five international conferences
  \item
    Six publications, two as lead author and four as co-author 
\end{rSubsection}

\begin{rSubsection}{Pulsars as a Probe of Magnetic Fields and Plasma Physics}{2012-14}{National Radio Astronomy Observatory}{Charlottesville VA}
  \item
    Full-time research as a summer intern 
  \item 
    Wrote \href{https://github.com/annayqho/ter5-RMs}{open-source code} in Python to measure the 
    galaxy's magnetic field using radio data
  \item Wrote a successful application for observing time on the 
    Very Large Array in New Mexico
  \item Presented work as a poster at two
    American Astronomical Society conferences,
    and at the National Science Foundation (NSF) Conference of 
    Research Experiences for Undergraduates
\end{rSubsection}

\end{rSection}

%------------------------------------------------------------------------------
%   HONORS AND AWARDS
%------------------------------------------------------------------------------

\begin{rSection}{Honors and Awards}

\begin{itemize}
\item
National Science Foundation Graduate Research Fellowship \hfill {\em 2014-19}
\item
Fulbright Scholarship \hfill {\em 2014-15}
\item
  Karl Taylor Compton Prize, MIT: \hfill {\em 2014} \\
``the highest awards presented by the Institute\ldots
in recognition of excellent achievements in citizenship 
and devotion to the welfare of MIT\ldots
sustained over a significant number of years.'' 
\item
First Place, Dewitt Wallace Prize for Science Writing for the Public, 
MIT \hfill {\em 2013} 
\end{itemize}

\end{rSection}

%------------------------------------------------------------------------------
%   TEACHING AND OUTREACH EXPERIENCE SECTION
%------------------------------------------------------------------------------

\begin{rSection}{Teaching and Public Outreach}

  \begin{rSubsection}{Caltech Astronomy Public Outreach}{Feb 2015-present}
    {}{Pasadena, CA}
  \item Run outreach evenings, give public talks, answer visitors' questions,
    facilitate telescope viewing
  \end{rSubsection}

  \begin{rSubsection}{Teaching Assistant, Caltech and MIT} 
    {2013-present}{}{MIT \& Caltech}
  \item Grade student work, hold office hours, LaTeX lecture notes
    for undergraduate and graduate courses
  \end{rSubsection}

  \begin{rSubsection}{Teacher for K-12 Educational Programs}
    {2010-2017}{}{}
  \item Educational Studies Program at MIT,
    Institute for Educational Advancement in Pasadena:
    designed and taught science
    classes for over 500 K-12 students.
  \item MIT/Teach For America:
    teaching assistant on
    the Pueblo Pintado Navajo Reservation in New Mexico. 
    Created new modules for the math curriculum, 
    tutored and mentored students.
  \end{rSubsection}

  \begin{rSubsection}{AAS Astronomy Ambassadors Workshop}{January 2014}
  {American Astronomical Society 223rd Meeting}{National Harbor, MD}
  \item One of 30 astronomers selected for professional development workshop on effectively communicating science with the public
  \end{rSubsection}

  \begin{rSubsection}{McCormick Public Observatory}{Summer 2012, Summer 2013}
    {Public speaker and volunteer}{Charlottesville, VA}
  \item Ran a volunteering program for National Radio Astronomy Observatory summer interns
  \end{rSubsection}

\end{rSection}

\begin{rSection}{Writing}

  \begin{rSubsection}
    {Haus der Astronomie}{Sept 2014-July 2015}
    {Center for Astronomy Education and Outreach}{Heidelberg, Germany}
  \item Wrote \href{http://www.mpia.de/news/science/2015-03-biosignatures}
    {a press release} for the Max Planck Institute for Astronomy
  \item Wrote \href{http://www.universetoday.com/120820/distant-stellar-nurseries-this-time-in-high-definition/}
    {a blog post} for the UniverseToday news site
  \end{rSubsection}

  \begin{rSubsection}{MIT Admissions Blogger}{Fall 2010-Spring 2014}
    {MIT Admissions Office}{Cambridge, MA}
  \item Wrote \href{http://mitadmissions.org/blogs/profile/annayq}{weekly entries} read by over 7,000 daily, corresponded with students through webcasts
  \end{rSubsection}

\end{rSection}


\begin{rSection}{Leadership}

  \begin{rSubsection}{Science-Engineering-Technology Congressional Visits Day (SETCVD)}{March 2014}{}{Washington, DC}
  \item The only undergraduate student selected to represent the American Astronomical Society
  \item On Capitol Hill, two days of training and workshops about the federal budgeting process 
  \item Set up and led meetings with Congressional Staff to advocate for federal funding for research 
  \end{rSubsection}

\begin{rSubsection}{MIT Chancellor's Student Leadership Summit}{September 2013}{}{}
\item One of 30 undergraduate students nominated to attend this annual one-day summit
\item Received training on the MIT administrative process and effectively advocating for student issues 
\end{rSubsection}

\begin{rSubsection}{MIT Committee on the Undergraduate Program (CUP)}{2013-14}{}{}
\item One of four student representatives, CUP exercises oversight for undergraduate academic program
\end{rSubsection}

\begin{rSubsection}{MIT Dormitory President}{2013}{}{}
\item Supervised committees and officers, represented dormitory to Undergraduate Association and Dormitory Council, ran events to provide a residential support system for students
\end{rSubsection}

\end{rSection}

\begin{rSection}{Technical Skills}
\item Python since 2012, one year of Java, one year of MATLAB
\item Comfortable in Linux, git, Vim
\end{rSection}

\end{document}
