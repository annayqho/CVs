%%%%%%%%%%%%%%%%%%%%%%%%%%%%%%%%%%%%%%%%%
% Medium Length Professional CV
% LaTeX Template
% Version 2.0 (8/5/13)
%
% This template has been downloaded from:
% http://www.LaTeXTemplates.com
%
% Original author:
% Trey Hunner (http://www.treyhunner.com/)
%
% Important note:
% This template requires the resume.cls file to be in the same directory as the
% .tex file. The resume.cls file provides the resume style used for structuring the
% document.
%
%%%%%%%%%%%%%%%%%%%%%%%%%%%%%%%%%%%%%%%%%

%----------------------------------------------------------------------------------------
%	PACKAGES AND OTHER DOCUMENT CONFIGURATIONS
%----------------------------------------------------------------------------------------

\documentclass{resume} % Use the custom resume.cls style
\usepackage{color,hyperref}
\definecolor{darkblue}{rgb}{0.0,0.0,0.8}
\hypersetup{colorlinks,breaklinks,linkcolor=darkblue,urlcolor=darkblue,anchorcolor=darkblue,citecolor=darkblue}
\usepackage[left=0.75in,top=0.6in,right=0.75in,bottom=0.6in]{geometry} % Document margins

\name{Anna Ho} % Your name
%\address{Max-Planck-Institut f{\"u}r Astro¨nomie \\ K{\"o}nigstuhl 17, 69117 Heidelberg, Germany} % Your address
\address{annayqho@gmail.edu \\ \href{annaho.net}{annaho.net}} % Your phone number and email

\begin{document}

%----------------------------------------------------------------------------------------
%	EDUCATION SECTION
%----------------------------------------------------------------------------------------

\begin{rSection}{Education}

{\bf California Institute of Technology (Caltech)}, Pasadena, CA
\begin{itemize}
\item Ph.D., Astrophysics \hfill {expected June 2020} 
\end{itemize}

{\bf Massachusetts Institute of Technology (MIT)}, Cambridge, MA
\begin{itemize}
\item B.S., Physics \hfill {received June 2014} 
\end{itemize}

\end{rSection}

%----------------------------------------------------------------------------------------
%	EMPLOYMENT SECTION
%----------------------------------------------------------------------------------------

\begin{rSection}{Employment}

\begin{rSubsection}{Fulbright Scholar}{September 2014-July 2015}{German-American Fulbright Commission}{Heidelberg, Germany}
\item Astrophysics research at the Max Planck Institute for Astronomy (MPIA) in Heidelberg, Germany
\item Public outreach and science writing at the MPIA-affiliated House of Astronomy
\end{rSubsection}

\begin{rSubsection}{Student Blogger}{August 2010-June 2014}{MIT Admissions Office}{Cambridge, MA}
\item Wrote \href{http://mitadmissions.org/blogs/author/annayq/archives}{weekly blog posts} on the MIT admissions website, read by over 7,000 visitors daily
\item Blog posts featured by the \href{https://aas.org/posts/blog/2014/04/congressional-visits-day-undergraduate-perspective}{American Astronomical Society} and the \href{http://web.mit.edu/mitpsc/pressroom/stories/ho.html}{MIT Public Service Center} %, and \href{https://www.reddit.com/r/todayilearned/comments/20ac4w/til_that_before_neil_degrasse_tyson_went_on_the/}{reddit}
\item Corresponded with prospective students through e-mail and webcasts
\end{rSubsection}

\begin{rSubsection}{Undergraduate Researcher}{Summer 2012, Summer 2013}{National Radio Astronomy Observatory}{Charlottesville, VA}
\item NSF Research Experiences for Undergraduates (REU) Program
\item Presented my work at the 2013 and 2014 American Astronomical Society Annual Winter Conference as well as at the 2013 Council on Undergraduate Research Conference
\end{rSubsection}

\begin{rSubsection}{Course Assistant}{Spring 2013}{MIT Physics Department}{Cambridge, MA}
\item Wrote lecture notes in LaTeX for the undergraduate Quantum I and Quantum II courses
\item Graded weekly problem sets for the undergraduate Quantum I course
\end{rSubsection}

\end{rSection}

%----------------------------------------------------------------------------------------
%	HONORS AND AWARDS
%----------------------------------------------------------------------------------------

\begin{rSection}{Fellowships and Awards}

\begin{itemize}
\item
National Science Foundation Graduate Research Fellowship \hfill {\em 2014-19}
\item
Fulbright Scholarship \hfill {\em 2014-15}
\item
Ida M. Green Fellowship, MIT \hfill {\em 2014} 
\item
Ford Foundation Fellowship, Honorable Mention \hfill {\em 2014} 
\item
Karl Taylor Compton Prize, MIT \hfill {\em 2014} 
\item
American Astronomical Society Chambliss Astronomy Achievement Student Awards, \hfill {\em 2014}
Honorable Mention 
\item
First Place, Dewitt Wallace Prize for Science Writing for the Public, MIT \hfill {\em 2013} 
\item
Burchard Scholars Program, MIT \hfill {\em 2012} \\
\end{itemize}

\end{rSection}

%----------------------------------------------------------------------------------------
%	TEACHING, OUTREACH, PUBLIC POLICY EXPERIENCE SECTION
%----------------------------------------------------------------------------------------

\begin{rSection}{Teaching, Public Outreach, and Science Policy}

\begin{rSubsection}{Congressional Visits Day}{March 2014}{American Astronomical Society Representative}{Washington, DC}
\item Served as one of 15 representatives of the American Astronomical Society on Capitol Hill for the annual Science-Engineering-Technology Working Group Congressional Visits Day
\item Two days of training and workshops about the federal budgeting process, how to engage with congressional staff, advocating for science funding, and particular key issues in astronomy
\item Set up and led meetings with Massachusetts Congressional Staff to advocate for federal funding for scientific research and undergraduate research programs
\end{rSubsection}

\begin{rSubsection}{AAS Astronomy Ambassadors Workshop}{January 2014}{American Astronomical Society 223rd Meeting}{National Harbor, MD}
\item One of 30 early-career astronomers selected for a two-day professional development workshop on effectively communicating with students and the public
\end{rSubsection}

\begin{rSubsection}{MIT Educational Studies Program}{Fall 2010-Spring 2014}{Teacher}{Cambridge, MA}
\item Designed and taught 12 different science classes for over 500 middle- and high-school students, ranging from stand-alone lectures to eight-week-long courses
\end{rSubsection}

\begin{rSubsection}{McCormick Public Observatory}{Summer 2012, Summer 2013}{Public speaker and volunteer}{Charlottesville, VA}
\item Organized a volunteering program for National Radio Astronomy Observatory summer students
\item Gave regular public talks on astronomy research at the observatory
\end{rSubsection}

\begin{rSubsection}{MIT Four Weeks For America Program, Teach For America}{January 2011}{Teaching Assistant}{Pueblo Pintado Navajo Reservation, NM}
\item Created new modules for the high school math curriculum
\item Tutored and mentored students
\end{rSubsection}

\end{rSection}

%----------------------------------------------------------------------------------------
%	CAMPUS LEADERSHIP SECTION
%----------------------------------------------------------------------------------------

\begin{rSection}{Campus Leadership}

\begin{rSubsection}{MIT Chancellor's Student Leadership Summit}{September 2013}{}{}
\item One of 30 undergraduates nominated to attend this annual one-day summit
\item Training on the MIT administrative process, workshops on effectively advocating for student issues 
\end{rSubsection}

\begin{rSubsection}{MIT Committee on the Undergraduate Program (CUP)}{2013-14}{}{}
\item Served as one of four student representatives on the CUP, which exercises oversight for the undergraduate academic program
\end{rSubsection}

\begin{rSubsection}{MIT New House Dormitory President}{2013}{}{}
\item Supervised committees and officers, represented New House to the Undergraduate Association and Dormitory Council
\end{rSubsection}

\begin{rSubsection}{MIT Associate Advisor}{Fall 2012-Spring 2014}{}{}
\item Served as an associate advisor to help freshmen navigate their first year at MIT
\item Represented New House on the Associate Advising Steering Committee, organized and ran events in New House to provide a residential support system
\end{rSubsection}

\end{rSection}

\end{document}
