%------------------------------------------------------------------------------
%   PACKAGES AND OTHER DOCUMENT CONFIGURATIONS
%------------------------------------------------------------------------------

\documentclass{resume} % Use the custom resume.cls style

% document margins
\usepackage[left=0.75in,top=0.6in,right=0.75in,bottom=0.6in]{geometry}
\usepackage{hyperref}

\name{Anna Ho} % Your name
\address{ah@astro.caltech.edu} % Your phone number and email
\address{Citizenship: US, UK}
\address{Caltech, MC 249-17, 1200 E. California Blvd., Pasadena CA 91125}

\begin{document}

%------------------------------------------------------------------------------
%   EDUCATION SECTION
%------------------------------------------------------------------------------

\begin{rSection}{Education}

{\bf California Institute of Technology (Caltech)}, Pasadena CA
\begin{itemize}
\item Ph.D., Astrophysics, 
  Advisor: Prof. Shrinivas Kulkarni \hfill expected {\em June 2020}
\end{itemize}

\begin{itemize}
  \item M.S., Astrophysics \hfill expected {\em June 2017}
\end{itemize}

{\bf Massachusetts Institute of Technology (MIT)}, Cambridge MA
\begin{itemize}
\item B.S., Physics \hfill {\em June 2014}
\end{itemize}

\end{rSection}


%------------------------------------------------------------------------------
%	RESEARCH SECTION
%------------------------------------------------------------------------------

\begin{rSection}{Research Projects}
\emph{The Cannon:} Data-Driven Modeling for Large Datasets of Spectra
\hfill 2014-17
\begin{itemize}
  \item
    As a graduate student at Caltech and a Fulbright Scholar in Germany,
    wrote open-source code in Python to model large datasets of spectra
  \item
    On invitation, designed and taught a two-day workshop on using Python to
    model spectra at Gemini Observatory in La Serena, Chile. 
    Audience include graduate students, post docs, and faculty from
    around Chile.
  \item
    Presented work as a talk at five international conferences
  \item
    Presented work as lead author on two publications and 
    co-author on four publications
\end{itemize}

Pulsars as a Probe of Magnetic Fields and Plasma Physics
\hfill 2012-14
\begin{itemize}
  \item As a summer intern at the National Radio Astronomy Observatory 
    in Charlottesville, VA, wrote open-source code in Python to measure the 
    galaxy's magnetic field using radio telescope data
  \item Wrote a successful application for observing time on the 
    Very Large Array in New Mexico
  \item Presented work as a poster at two
    American Astronomical Society conferences,
    and at the National Science Foundation (NSF) Conference of 
    Research Experiences for Undergraduates
\end{itemize}

\end{rSection}

%------------------------------------------------------------------------------
%   HONORS AND AWARDS
%------------------------------------------------------------------------------

\begin{rSection}{Honors and Awards}

\begin{itemize}
\item
National Science Foundation Graduate Research Fellowship \hfill {\em 2014-19}
\item
Fulbright Scholarship \hfill {\em 2014-15}
\item
Ida M. Green Fellowship, MIT \hfill {\em 2014} 
\item
  Karl Taylor Compton Prize, MIT: \hfill {\em 2014} \\
``the highest awards presented by the Institute\ldots
in recognition of excellent achievements in citizenship 
and devotion to the welfare of MIT. 
They reflect outstanding contributions to the MIT community as a whole, 
sustained over a significant number of years.'' 
\item
First Place, Dewitt Wallace Prize for Science Writing for the Public, 
MIT \hfill {\em 2013} 
\item
Burchard Scholars Program, MIT \hfill {\em 2012} \\
\end{itemize}

\end{rSection}

%------------------------------------------------------------------------------
%   TEACHING AND OUTREACH EXPERIENCE SECTION
%------------------------------------------------------------------------------

\begin{rSection}{Teaching and Public Outreach}

  \begin{rSubsection}{Caltech Astronomy Public Outreach}{Feb 2015-present}
    {}{Pasadena, CA}
  \item Run outreach evenings, give public talks, answer visitors' questions,
    facilitate telescope viewing
  \end{rSubsection}

  \begin{rSubsection}{Teaching Assistant} 
    {2013-present}{}{MIT \& Caltech}
  \item Caltech Astrophysics Dept: grade student work, hold office hours 
    for UG and graduate courses
  \item MIT Physics Dept: Transcribed lecture notes and graded student work
    for UG quantum mechanics
  \end{rSubsection}

  \begin{rSubsection}{Teacher for K-12 Educational Programs}
    {2010-2017}{}{}
  \item Through the Educational Studies Program at MIT and the 
    Institute for Educational Advancement in Pasadena, 
    I have designed and taught science
    classes for over 500 K-12 students.
  \item Through the MIT/Teach For America Four Weeks For America Program,
    I served as a teaching assistant for one month on a high school in
    the Pueblo Pintado Navajo Reservation in New Mexico. I created
    new modules for the math curriculum, and tutored and mentored students.
  \end{rSubsection}

  \begin{rSubsection}
    {Haus der Astronomie}{Sept 2014-July 2015}
    {Center for Astronomy Education and Outreach}{Heidelberg, Germany}
  \item Organized and taught a cosmology workshop for high school students
  \item Wrote \href{http://www.mpia.de/news/science/2015-03-biosignatures}
    {a press release} for the Max Planck Institute for Astronomy
  \item Wrote \href{http://www.universetoday.com/120820/distant-stellar-nurseries-this-time-in-high-definition/}
    {a blog post} for the UniverseToday news site
  \end{rSubsection}

  \begin{rSubsection}{AAS Astronomy Ambassadors Workshop}{January 2014}
  {American Astronomical Society 223rd Meeting}{National Harbor, MD}
  \item Selected for a two-day professional development workshop on
    effectively communicating science with the public
  \end{rSubsection}

  \begin{rSubsection}{MIT Admissions Blogger}{Fall 2010-Spring 2014}
    {MIT Admissions Office}{Cambridge, MA}
  \item Wrote weekly entries about MIT life, read by over 7,000 people daily
  \item Corresponded with prospective students through e-mail and webcasts
  \end{rSubsection}

  \begin{rSubsection}{McCormick Public Observatory}{Summer 2012, Summer 2013}
    {Public speaker and volunteer}{Charlottesville, VA}
  \item Organized a volunteering program for National Radio Astronomy Observatory 
    summer students
  \item Gave regular public talks at the observatory
  \end{rSubsection}

\end{rSection}


\begin{rSection}{Leadership}

  \begin{rSubsection}{Science-Engineering-Technology Congressional Visits Day (SETCVD)}{March 2014}{}{Washington, DC}
  \item The only undergraduate student among 15 graduate students, post docs, and professors selected to represent the American Astronomical Society on Capitol Hill for the annual SETCVD
  \item Two days of training and workshops about the federal budgeting process 
  \item Set up and led meetings with Congressional Staff to advocate for federal funding for scientific research and undergraduate research programs
  \end{rSubsection}

\begin{rSubsection}{MIT Chancellor's Student Leadership Summit}{September 2013}{}{}
\item One of 30 undergraduate students nominated to attend this annual one-day summit
\item Received training on the MIT administrative process and effectively advocating for student issues 
\end{rSubsection}

\begin{rSubsection}{MIT Committee on the Undergraduate Program (CUP)}{2013-14}{}{}
\item Served as one of four student representatives on the CUP, which exercises oversight for the undergraduate academic program
\end{rSubsection}

\begin{rSubsection}{MIT New House Dormitory President}{2013}{}{}
\item Supervised committees and officers, represented New House to the Undergraduate Association and Dormitory Council
\end{rSubsection}

\begin{rSubsection}{MIT Associate Advisor}{Fall 2012-Spring 2014}{}{}
\item Served as an associate advisor to help freshmen navigate their first year at MIT
\item Represented New House on the Associate Advising Steering Committee, organized and ran events in New House to provide a residential support system
\end{rSubsection}

\end{rSection}

\end{document}
